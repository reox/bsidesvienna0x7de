\documentclass[12pt,a4paper]{beamer}
\usepackage[utf8]{inputenc}
\usepackage[german]{babel}
\usepackage{amsmath}
\usepackage{amsfonts}
\usepackage{amssymb}
\usepackage{xcolor}
\usepackage{listings}
\usepackage{textcomp}
\usepackage{dirtree}
\usepackage{dialogue}

\definecolor{shadecolor}{RGB}{0,0,0}
\definecolor{textcolor}{RGB}{255,255,255}

% Define a Macro to create Image Overlays
\newcommand{\mybox}[1]{\par\noindent\colorbox{shadecolor}
{\color{textcolor}\parbox{\dimexpr\textwidth-2\fboxsep\relax}{\fontsize{3em}{3.5em}\selectfont\textbf{{#1}}}}}

\lstset{
    frame=single,
    breaklines=true,
    postbreak=\raisebox{0ex}[0ex][0ex]{\ensuremath{\color{red}\hookrightarrow\space}},
}


\usepackage{tikz}
\usetheme{CambridgeUS}
\setbeamertemplate{navigation symbols}{}%remove navigation symbols

% Presentation Metadata
\author{Sebastian Bachmann \& Tibor Éliás}
\title{How we hacked Online Banking Malware}
\date{22. November 2014}

\begin{document}

\begin{frame}
    \maketitle
    \centering
    B-Sides Vienna
\end{frame}


\section{About Us}
\begin{frame}
	\frametitle{About: Sebastian Bachmann \& Tibor Éliás}
	\begin{itemize}
		\item Mobile Malware Analyst at IKARUS since 2012 / 2013
		\item Analyse Android Malware
		\item Research
		\item Analysis of Incidents
	\end{itemize}
\end{frame}

\section{About this talk}
\begin{frame}
	\frametitle{What is this all about?}
	\begin{enumerate}
	
		\item Customer Incident: Online Banking Fraud
		\item How we totally fucked up analysis
		\item How we recovered
		\item ... and of course: what we learned!
	\end{enumerate}
\end{frame}


\section{First Analysis}

\begin{frame}
	\frametitle{The incident}
	
	\begin{itemize}
		\item Online Banking Trojan detected on PC
		\item Suspicion of mobile component used
		\item Device: Samsung Galaxy Nexus (i9250), Android 4.1
		\item And of course: friday afternoon
	\end{itemize}

\end{frame}

\begin{frame}
\frametitle{Start the Analysis}
\begin{itemize}
	\item[\textbf{\color{green}+}] No ADB enabled
	\item[\textbf{\color{green}+}] No suspicious App icons shown
	\item[\textbf{\color{green}+}] Device is not rooted
	\item[\textbf{\color{red}-}] Unknown sources enabled 
	\item[\textbf{\color{red}-}] App lists shows a suspicious app
	\item[\textbf{\color{red}-}] We already knew that the device was compromised
\end{itemize}
\end{frame}

\begin{frame}[fragile]
	\frametitle{Next steps}
	
	\begin{itemize}
		\item Enable ADB
		\item Pull all installed APKs from device
	
	\begin{lstlisting}[language=bash]
	for app in $(adb shell pm list packages -f | cut -d ':' -f 2 | cut -d '=' -f 1); do 
	DIR=$(dirname $app | tr '/' '_'); 
	[[ -d $DIR ]] || mkdir $DIR && :; 
	adb pull $app $DIR/; done
	\end{lstlisting}

	\item found \texttt{com.certificate-1.apk}
	\end{itemize}	
\end{frame}

\newcommand{\certificateIcon}{\includegraphics[keepaspectratio=true,height=0.8cm]{images/icon.png} \texttt{com.certificate-1.apk}}

\begin{frame}
\frametitle{\certificateIcon}
	
	\begin{itemize}
		\item MD5: \texttt{a10fae2ad515b4b76ad950ea5ef76f72}
		\item Package Name: \texttt{com.certificate}
		\item Two Activity
		\item One Service
		\item Three Receivers
		\item 15+ positive results on VirusTotal
		\item Already known as ,,Hesperbot''
	\end{itemize}

\end{frame}

\begin{frame}

\frametitle{\certificateIcon}
\resizebox{0.8\textwidth}{0.4\textheight}{
\parbox{\textwidth}{
\dirtree{%
.1 com.certificate-1.apk. 
.2 META-INF. 
.3 CERT.SF. 
.3 MANIFEST.MF. 
.3 CERT.RSA. 
.2 resources.asrc. 
.2 classes.dex\DTcomment{Dalvik Executeable}. 
.2 AndroidManifest.xml. 
.2 assets. 
.3 spy.db\DTcomment{SQLite Database}. 
.2 res. 
.3 xml. 
.4 device\_admin\_policies.xml. 
.3 layout. 
.4 main.xml\DTcomment{Layout File for MainActivity}. 
.3 drawable. 
.4 icon.png. 
}% End dirtree
}
}
\end{frame}


\begin{frame}
\frametitle{\certificateIcon}
	\begin{itemize}
\item android.permission.SEND\_SMS
\item android.permission.INTERNET
\item android.permission.RECEIVE\_WAP\_PUSH
\item android.permission.WRITE\_SMS
\item android.permission.PROCESS\_OUTGOING\_CALLS
\item android.permission.GET\_TASKS
\item android.permission.RECEIVE\_SMS
\item android.permission.READ\_CONTACTS
\item android.permission.RECEIVE\_MMS
\item android.permission.WRITE\_EXTERNAL\_STORAGE
\item android.permission.READ\_SMS
\item android.permission.READ\_LOGS
\item android.permission.RECEIVE\_BOOT\_COMPLETED
\item android.permission.KILL\_BACKGROUND\_PROCESSES
	\end{itemize}
\end{frame}


{
\usebackgroundtemplate{\includegraphics[height=\paperheight]{images/androidmalware.jpg}}
\begin{frame}[plain]

\raisebox{-10em}{\mybox{Malware found...}}
\footnotetext{\color{white}Image (CC BY 2.0) from: https://flic.kr/p/cuZZUY}
\end{frame}
}


\section{How we fucked up}

\begin{frame}
\frametitle{Meanwhile...}
\begin{dialogue}
\speak{Sebastian} Okay, weekend starts soon so I better remove that thing from the device so we can send it back...

\speak{Tibor} I will start analysis of the sample then and write the report.

\speak{Sebastian} Do you need anything from the device before I remove the malware?

\speak{Tibor} I don't think so...
\end{dialogue}
\end{frame}


\begin{frame}
\frametitle{Removal...}
\textit{Live Demo here... hopefully}
\end{frame}




\section{Shock!}
\begin{frame}
	\frametitle{Meanwhile...}
	\begin{dialogue}
	\speak{Sebastian} Ahh what?
	\speak{Tibor} What was that?
	\speak{Sebastian} I don't know... What was the device PIN again? \direct{tries the PIN...}
	\speak{Tibor} Looks like you just locked the device!
	\speak{Sebastian} Uh oh...
	\end{dialogue}
\end{frame}

{
\usebackgroundtemplate{\includegraphics[height=\paperheight]{images/fuuu.jpg}}
\begin{frame}[plain]
%\mybox{FFFFFUUUUU!!!}
\end{frame}
}

\section{Reversing}

\begin{frame}[fragile]
\frametitle{A closer look at the Malware}
What's happening on DeviceAdmin onDisableRequest?

\begin{lstlisting}[language=java,basicstyle=\smaller,stringstyle=\color{orange},identifierstyle=\color{blue}]
if (com.certificate.Cache.getInstance().isContainsSetting("rCode")) {
  String v14 = com.certificate.Util.EncodeThis("uninstall").replace(" ", "");
  v13 = v14.substring(0, (v14.length() - 1));
}
Object v3 = p9.getSystemService("device_policy");
if ((com.certificate.ModuleAdminReceiver.IS_SELF_DEACTIVATION) && (v13.length() > 0)) {
  v3.resetPassword(v13, 0);
  com.certificate.ModuleAdminReceiver.IS_UNINSTALLING = 1;
  v3.lockNow();
}
\end{lstlisting}

\end{frame}

\begin{frame}
\frametitle{A closer look at the Malware}

\begin{itemize}
	\item \texttt{EncodeThis} uses RC5\\
	Blocksize 32bit, Cipher Length 64bit and 12 Rounds
	\item The Cipher is initialised from \texttt{rCode}
	\item \texttt{rCode} (=Response Code) is set on Malware Activation
\end{itemize}

\end{frame}


\begin{frame}
\frametitle{A closer look at the Malware}

\centering
\includegraphics[height=0.6\textheight]{images/activation.png}
\end{frame}

\begin{frame}
\frametitle{Response Code Generation}
\centering
\includegraphics[height=0.8\textheight]{images/code_generation.png}
\end{frame}

\begin{frame}

\mybox{Activiation Code\\is unknown...}
\newline
\newline,
... and there is no chance to get it from anywhere

\end{frame}

{
\usebackgroundtemplate{\includegraphics[height=\paperheight]{images/deeper.jpg}}
\begin{frame}[plain]
\raisebox{16em}{\mybox{We need\\to go deeper}}

\footnotetext{\color{white}Image (CC-PD) from: http://goo.gl/WxHtjp}
\end{frame}
}

\begin{frame}
\frametitle{Open Questions}
\begin{itemize}
	\item How was the DeviceAdmin enabled on the device?
	\item Was or is there any communication with the Botmaster?
	\item Can we get the Response Code out of the device?
	\item Is there a way to bruteforce the key?
	\item Is there another trap?
\end{itemize}
\end{frame}

\begin{frame}
\frametitle{Bruteforce the Key?}
\begin{itemize}
	\item Only 10k different \texttt{rCode}s
	\item Every uninstall code is 25 chars
	\item 30s lock after 5 wrong logins
	\item 5s to enter 5 codes + 30s pause: 48h in average
	\item + the time to generate all codes first
\end{itemize}

\textbf{Answer}: probably not

\end{frame}

\begin{frame}
\frametitle{Can we get the Response Code out of the device?}
\begin{itemize}
	\item \texttt{cert.db} is in the Apps userdata storage
	\item These files are not \texttt{RW} for shell/adb user
	\item No Root Access on the Device
	\item Root the Device by Bootloader would delete all data (Bootloader was still locked)
\end{itemize}

\textbf{Answer}: No, we can not
\end{frame}

\begin{frame}
\frametitle{How was the DeviceAdmin enabled?}

\begin{itemize}
	\item After starting MainActivity start a Service
	\item Service invokes Activity for DeviceAdmin Request
	\item Service checks if Admin is set
	\item DeviceAdmin Activity calls Utility Class
	\item Utility Class creates a timer and shows the Request every 3s
\end{itemize}

\textbf{Answer}: The User clicked in Panic on the Activate Button

\end{frame}

\begin{frame}[fragile]
\frametitle{DeviceAdmin Request}
\begin{lstlisting}[language=java,basicstyle=\smaller,stringstyle=\color{orange},identifierstyle=\color{blue}]
java.util.Timer v32 = new java.util.Timer();
android.content.Intent v38 = new android.content.Intent("android.app.action.ADD_DEVICE_ADMIN");
v38.putExtra("android.app.extra.DEVICE_ADMIN", v30);
v38.putExtra("android.app.extra.ADD_EXPLANATION", "Allow to protect uninstallation of app");
v32.scheduleAtFixedRate(new com.certificate.Util$3(v1, v30, v32, p15, v38), ((long) v12), 3000.0);
\end{lstlisting}
Timer Creation and DeviceAdmin Request

\end{frame}

\end{document}